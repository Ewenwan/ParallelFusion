% #BIBTEX /Library/Tex/texbin/bibtex eccv2016
% last updated in April 2002 by Antje Endemann
% Based on CVPR 07 and LNCS, with modifications by DAF, AZ and elle, 2008 and AA, 2010, and CC, 2011; TT, 2014; AAS, 2016

\documentclass[runningheads]{llncs}
\usepackage{graphicx}
\usepackage{subcaption}
\captionsetup{compatibility=false}
\usepackage{amsmath,amssymb} % define this before the line numbering.
\usepackage{ruler}
\usepackage{color}
% \usepackage{tensor}
\usepackage{algorithm}
\usepackage{algpseudocode}
\usepackage{ulem}
\usepackage{url}
\usepackage{cite}

\usepackage[width=122mm,left=12mm,paperwidth=146mm,height=193mm,top=12mm,paperheight=217mm]{geometry}


\algnewcommand\algorithmicforeach{\textbf{for each}}
\algdef{S}[FOR]{ForEach}[1]{\algorithmicforeach\ #1\ \algorithmicdo}

\newcommand{\mysubsubsection}[1]{\vspace{0.2cm} \noindent
  \underline{{\bf #1}} \vspace{0.1cm}}
\newcommand{\mysubsubsubsection}[1]{\vspace{0.1cm} \noindent {\bf #1}:}

% \newcommand{\pushmeet}[1]{\textcolor{blue}{pushmeet:{#1}}}
% \newcommand{\yasu}[1]{\textcolor{magenta}{[yasu: {#1}]}}
\newcommand{\chen}[1]{\textcolor{cyan}{[chen: {#1}]}}
% \newcommand{\hang}[1]{\textcolor{green}{[hang: {#1}]}}
% \newcommand{\pushmeet}[1]{}
% \newcommand{\chen}[1]{}
% \newcommand{\yasu}[1]{}
% \newcommand{\hang}[1]{}

\ifx\pdfoptionalwaysusepdfpagebox\relax\else
\pdfoptionalwaysusepdfpagebox5
\fi

\begin{document}
% \renewcommand\thelinenumber{\color[rgb]{0.2,0.5,0.8}\normalfont\sffamily\scriptsize\arabic{linenumber}\color[rgb]{0,0,0}}
% \renewcommand\makeLineNumber {\hss\thelinenumber\ \hspace{6mm} \rlap{\hskip\textwidth\ \hspace{6.5mm}\thelinenumber}}
% \linenumbers
\pagestyle{headings}
\mainmatter
\def\ECCV16SubNumber{1569}  % Insert your submission number here

% \title{Parallelizing Hypothesis Fusion for Rapid MAP inference}
\title{Multi-way Particle Swarm Fusion}

\maketitle
\thispagestyle{empty}


%%%%%%%%% BODY TEXT - ENTER YOUR RESPONSE BELOW
We thank the reviewers for insightful comments. Major comments are
the limitation of using only 4 threads and the problem specific choice of architectures. We first address these major comments.

\vspace{0.15cm}
\noindent {\bf The lack of technical contribution:} We
believe that this paper proposes an effective parallel
framework which, based on the novel concept of solution
sharing and multi-way fusion, generalizes several existing
methods. With simple parameter tuning, our framework can be
specialized as an effective parallel architecture for any
given MRF inference tasks. We will also make our framework implementation public, so that researchers can utilize maximum parallelism by simply providing the energy function and solution proposals. So the
impact of this work would be huge.

\vspace{0.15cm}
\noindent {\bf The choice of architectures:} We agree with
R1 and R3 that the best fusing architecture is problem
specific. However, our parallel framework is general,
achieving all the architectures mentioned in the paper via
different parameter settings. This is indeed an advantage of
our framework because the best architecture can be found by
simple grid search over parameter space in the same
framework instead of trying various existing methods. And
existing methods, as the special cases of our framework, are
very unlikely to be the best architecture for any problem
settings. Note that there are only three parameters
($\alpha$, $\beta$, and the frequency of solution sharing)
controlling the parallel architecture, and all these three
are small integer numbers, so the grid search space can be
easily conducted. Note that in order to illustrate the
intuition behind solution sharing and multi-way fusion, we
only presented three special architecture of our framework
in the paper, namely SF-SS, SF-MF, and SF (despite of its
name, we use maximal solution sharing and multi-way fusion
in all our experiments). But to solve a
problem in practice, a grid search is required to find the best
parameter setting. We will add the discussion here to the
paper and provide codes for automatic parameter tuning.

\vspace{0.15cm} We believe that this paper proposes a novel
parallel framework which will benefit many researchers with
more effective parallel architectures over existing
methods. The best architecture for any given problem can be
found via simple grid search in the parameter space. Now, we
answer remaining questions from each reviewer. We will
clarify these points in the paper, too.

\vspace{0.15cm}
\noindent {\bf Reviewer 1}

\vspace{0.15cm}
\noindent $\bullet$ (The number of threads:) Note that the
complexity for each thread is independent of the number of
threads in total. With this nature, we can apply our
parallel scheme to any number of threads or even distributed
systems. To make comparison fair in our experiments, we only
use 4 threads to make sure all threads including the
additional monitor thread run on its own physical core. The
experiments focus more on the intuition behind solution
sharing and multi-way fusion instead of pursuing best performance. But in
practice, our current implementation allows the usage of any
number of threads. A future work would be implementing our
parallel scheme on distributed environments.

\noindent $\bullet$ (Compliants about Fig. 3:) Thanks for
pointing it out. This figure is indeed incomplete as we want
to stress speed difference and explain this intuition of
solution sharing. The complete figure is shown in Fig. 1
here. As the energy is submodular in this problem setting,
all the architectures can find the global minimum at the
end.

\noindent $\bullet$ (Line 416-417 statement contradicts SF-MF in Fig. 1:) As indicated by the red arrows, the last thread of SF-MF on the figure (far right) also explores solutions generated by other thread (which is not shown on the figure). We will make the figure more clear.

\noindent $\bullet$ (Underlining paragraph titles:) Thanks for pointing it out. We will change the format.

% \section{R2}
\vspace{0.15cm}
\noindent {\bf Reviewer 2}
\noindent $\bullet$ (Missing reference:) We thank R2 for
pointing out this paper, Simultaneous Fusion Moves for
3D­Label Stereo. However, to our understanding, the
multi-label fusion move proposed in this paper is different
with out multi-way fusion. Their multi-label fusion move
conduct one single TRW inference (with many iterations) to
fuse all proposals together, whereas our multi-way fusion
conduct one TRW inference to fuse only a subset of proposals
and repeat the fusion until convergence. When the energy
function becomes more complex or the number of proposals is
large, a single TRW over all proposals is very hard to
converge, but our multi-way fusion will still work by
repeatedly fusing small subset of proposals.

\chen{\noindent $\bullet$ (The benefit of multi-threading:)
  We admit that the multi-threading does not help the
  algorithm for some applications if run to
  convergence. However, it consistently benefits online
  optimization as it provides better results at each
  timestamp.}
  
\vspace{0.15cm}
\noindent {\bf Reviewer 3}

\noindent $\bullet$ (The intuition behind Swarm Fusion:) We
illustrate the benefit of utilizing solution sharing and
multi-way fusion through experiments and explain the
intution behind. The Swarm Fusion combines both ideas and is
a general scheme. These two ideas are intuitive and
effective. We are not aware of other relevant ideas.

\noindent $\bullet$ (Lack of exciting results:) We
appreciate the comment that some results we show are not
exciting. However, our experiments use extreme cases to
illustrate the idea of solution sharing and multi-way
fusion. While in practice, the best parameter setting is
unlikely to be any of these extreme cases and should be
found by a grid search over parameter space. However,
instead of achieving the best performance, we are more
excited about the effectivenss and applicability of our
framework, which makes it a handy tool for speeding up many
MRF inference problem. We will add relevant discussion in
the paper.

\noindent $\bullet$ (Dependent on proposal generations:) We
agree with R3 that the performance is highly dependent on
proposal generation scheme. This fact causes trouble for
fair comparison as some proposals are not applicable or
despite certain architectures. To make the comparison fair,
we often discard certain proposal generation schemes
in the cost of performance.

\noindent $\bullet$ (Proposal generation schemes from [15]:) \chen{I am not sure what this compliant is about as none of our experiments is about [15]}

\end{document}
