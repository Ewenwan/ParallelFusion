\section{Swarm Fusion Method}

Swarm Fusion is a natural multi-core extension of Fusion Move
Method. Take a multi-threading environment to explain our idea, which is
also applicable to other parallel computation environments such as cloud
computing. Every thread $T_i$ keeps track of a solution $S_i$, and
repeats updating $S_i$ by fusion ($i = 1, 2, \cdots N$). Two parameters
$\alpha_i$ and $\beta_i$ control the behavior of fusion steps in each
thread $T_i$. $\alpha_i$ specifies how many solution proposals are to be
generated from proposal generation schemes. Similarly, $\beta_i$
specifies how many solutions in the other threads are to be
fused. Suppose $N=4$, $\alpha_1 = 2$, and $\beta_1 = 3$. In this
configuration, the first thread $T_1$ fuses its current solution $S_1$,
two new proposal solutions (e.g., by randomly picking two proposal
generation schemes), and the current solutions $(S_2, S_3, S_4)$ in the
other three threads. This is a multi-way fusion with a non-submodular
energy in general, and message passing algorithm such as
TRW~\cite{kolmogorov} can be used. Sequential QPBO application is
another.


Our model is flexible. For example, a parallel fusion move algorithm by
Lempitsky et al.~\cite{viktor} is a special case of ours, where $\alpha_$

, and it is easy to see that 
Existing parallel fusion techniques are special 

Our model is flexible and 
