\subsection{Swarm Fusion optical flow}

Fusion Move was first introduced by Lempitsky et al.~\cite{viktor} to
solve optical flow problem.
%Optical flow was the problem, in which the Fusion Move method was first
%introduced by Lempitsky et al.~\cite{viktor}.
We copy their problem setting and fusion proposal schemes for our second
problem.  We again use test images in the Middlebury optical flow
benchmark~\cite{middlebury_optical_flow}.

\mysubsubsection{Competing methods} Fusion Move method in their paper is
the first natural contender. While they did not consider parallel
implementation, it is straightforward to combine the idea of parallel
alpha-expansion and Fusion Move. Then, the second competing method is
``parallel fusion move'', where each thread repeats generating a
solution proposal and fusing it to the current solution. Final solutions
from all the results are sequentially fused to obtain the final solution
as in the parallel alpha-expansion method. In Fig.~\ref{fig:model}, this
computational framework corresponds to ``parallel alpha-expansion'' with
the label solutions replaced by solution proposals. The last contender
is the mix of the hierarchical fusion and the fusion move method, dubbed
``hierarchical fusion method'', where they start from solution proposals
as opposed to constant labels.
%
The fusions are binary, for which we use QPBO.

\mysubsubsection{Swarm fusion architectures}
%The three swarm fusion models (at the bottom of Fig.~\ref{fig:model})
%are used to evaluate the effectiveness of our method.
The three swarm architectures in Fig.~\ref{fig:model} will be evaluated
against the competing methods. Note that we use the same set of proposal
generation schemes. The fusions are multi-way, for which we ues TRW-S.
\yasu{talk about our parameters $\alpha,\beta$}
% The full swarm fusion model (at the bottom right in
% Fig.~\ref{fig:model}) will be used to evaluate the effectiveness of our
% method. We will also try the other two Swarm fusion variants in
% Fig.~\ref{fig:model} with restrictions.
