\subsection{Swarm Fusion optical flow}

The second problem is an optical flow problem, where Lempitsky
introduced the fusion move algorithm~\cite{viktor}. We have used the
energy definitions and proposals generation schemes as described in the
paper. We again use test datasets from
Middlebury~\cite{middlebury_optical_flow}.

\mysubsubsection{Competing methods} The first baseline is the FM
algorithm exactly as given in the paper.  Although their paper did not
consider parallel implementation, it is straightfoward to apply the
concept of parallel alpha-expansion. Therefore, the second baseline is
the ``parallel fusion move'' method (replacing constant label solutions
with proposals inside ``parallel alpha-expansion'' in
Fig.~\ref{fig:model}).

\mysubsubsection{Swarm fusion architecture}
%The three swarm fusion models (at the bottom of Fig.~\ref{fig:model})
%are used to evaluate the effectiveness of our method.
The full swarm fusion model (at the bottom right in
Fig.~\ref{fig:model}) will be used to evaluate the effectiveness of our
method. We will also try the other two Swarm fusion variants in
Fig.~\ref{fig:model} with restrictions.
