\subsection{Swarm Fusion optical flow}

Fusion Move was first introduced by Lempitsky et al.~\cite{viktor} to
solve the optical flow problem.
%Optical flow was the problem, in which the Fusion Move method was first
%introduced by Lempitsky et al.~\cite{viktor}.
We copy their problem setting and proposal generation schemes for our
second problem.  We use test images in the Middlebury optical flow
benchmark~\cite{middlebury_optical_flow}.

\mysubsubsection{Competing methods} Fusion Move method in Lempitsky's
paper is the first natural contender. While they did not consider
parallel implementation, it is straightforward to combine the idea of
Parallel Alpha-Expansion and Fusion Move. Therefore, the second competing
method is ``Parallel Fusion Move (PFM)'', which is equivalent to
Parallel Alpha-Expansion with the constant label solutions replaced by
the solution proposals.
%each thread repeats generating
%a solution proposal and fusing it to the current solution. Solutions
%from all the results are sequentially fused to obtain the final solution
%as in the Parallel Alpha-Expansion method.
%
One problem of PFM is that infinite number of solution proposals can be
generated in their algorithm, and we do not know when we should stop and
perform the final sequential fusion. In our experiments, we manually
picked a time limit to initiate the final fusion.
%In Fig.~\ref{fig:model}, this computational framework corresponds to
%``parallel alpha-expansion'' with the label solutions replaced by
%solution proposals.
%
The last contender is the mix of the Hierarchical Fusion and the Fusion
Move method, dubbed ``Hierarchical Fusion Method'', where they start
from solution proposals as opposed to constant labels. One problem is
that we do not know how many solution proposals to generate to start
with. We have manually picked a good number (X \yasu{how many Hang?})
for the experiments.
%
The fusions are binary in these methods and we have used QPBO.

\mysubsubsection{Swarm fusion architectures}
%The three swarm fusion models (at the bottom of Fig.~\ref{fig:model})
%are used to evaluate the effectiveness of our method.
The three swarm architectures in Fig.~\ref{fig:model} (SFWSS, SFWMF, SF)
have been evaluated against the competing methods. Note that we have
used the same set of proposal generation schemes. The fusions are
multi-way, for which we have used TRW-S.
% The full swarm fusion model (at the bottom right in
% Fig.~\ref{fig:model}) will be used to evaluate the effectiveness of our
% method. We will also try the other two Swarm fusion variants in
% Fig.~\ref{fig:model} with restrictions.
We have chosen $\alpha=2$ ($\beta=0$) in SFWSS, the iteration of
($\alpha=1, \beta=0$) and ($\alpha=0, \beta=1$) in SFWMF, and
($\alpha=1, \beta=1$) for the standard SF.  The fusions are multi-way,
for which we use TRW-S.
