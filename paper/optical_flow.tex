\subsection{Swarm Fusion optical flow}

Fusion Move was first introduced by Lempitsky et al.~\cite{fusion_flow} to
solve the optical flow problem.
% Optical flow was the problem, in which the Fusion Move method was first
%introduced by Lempitsky et al.~\cite{viktor}.
We copy their problem setting and use images from the Middlebury optical flow
benchmark~\cite{middlebury_optical_flow}. We share similar proposal generation schemes with Lempitsky et al~\cite{fusion_flow}, but instead of the HS algorithm, we use more recent Farneback algorithm and change its pyramid levels from 1 to 5 and use either 3, 5 or 7 for parameter polyN. Besides the clustering idea, we add three simple proposal generation schemes based on the current solution as suggested in ~\cite{fusion_flow}. In \textit{shift proposal}, the flow field in the current solution is shifted in either x direction or y direction for either 1, 2 or 3 pixels. In \textit{stagger proposal}, the flow field is shifted by a vector randomly drawed from a Gaussian distribution. In \textit{disturb proposal}, each flow value in the field is indenpendantly shifted by a vector randomly drawed from a Gaussian distribution. We choose schemes randomly when generating proposals.

\mysubsubsection{Competing methods}

\noindent Fusion Move method in Lempitsky's paper is the first natural
contender. While they did not consider parallel implementation, it is
straightforward to combine the idea of Parallel Alpha Expansion and
Fusion Move. Therefore, the second competing method is ``Parallel Fusion
Move''(PFM), which is equivalent to Parallel Alpha Expansion with constant
label solutions replaced by solution proposals.
%each thread repeats generating
%a solution proposal and fusing it to the current solution. Solutions
%from all the results are sequentially fused to obtain the final solution
%as in the Parallel Alpha-Expansion method.
%
One problem of PFM is that infinite number of solution proposals can be
generated in their algorithm, and we do not know when to stop and
perform the final sequential fusion (See Parallel Alpha Expansion
architecture in Fig.~\ref{fig:model}). In our experiments, we manually
picked time limits to initiate the final fusion to make the comparisons
fair.
%In Fig.~\ref{fig:model}, this computational framework corresponds to
%``parallel alpha-expansion'' with the label solutions replaced by
%solution proposals.
%
The last contender is the mix of the Hierarchical Fusion and the Fusion
Move methods, dubbed ``Hierarchical Fusion Method''(HFM), where they
start from solution proposals as opposed to constant labels. One problem
of HFM is that we need to generate all proposals at the very beginning in order to build the fusion tree. This undermines the power of fusion move to dynamically generate solution proposals based on current solution. In our experiments, we have manually generate 250 proposals at the beginning. We first generate 25 proposals using LK and Farneback algorithms which are independant and generate other proposals based on these initial proposals). The fusions are binary in these methods. % and we have used QPBO.

\mysubsubsection{Swarm fusion architectures}

\noindent
%The three swarm fusion models (at the bottom of Fig.~\ref{fig:model})
%are used to evaluate the effectiveness of our method.
The three swarm architectures in Fig.~\ref{fig:model} (SF-MF, SF-SS, SF)
have been evaluated against the competing methods. For SF-MF, each thread repeats generating solution proposals ($\alpha=1, \beta=0$) for 4 iterations and fuses with one solution from others ($\alpha=0, \beta=1$) in one iteration. This pattern is repeated. For SF-SS, each thread always generates three solution proposals for fusion in each iteration ($\alpha=3$, $\beta=0$). For SF, the repeating pattern is ($\alpha=3, \beta=0$) for 4 iterations and ($\alpha=0, \beta=3$) for one iteration.

%
Note that we have used the same set of proposal generation schemes as
the competing methods. We have used TRW-S for multi-way fusion and QPBO
for binary fusion.
% , while our fusions are multi-way, for which we have used TRW-S.
%
% The full swarm fusion model (at the bottom right in
% Fig.~\ref{fig:model}) will be used to evaluate the effectiveness of our
% method. We will also try the other two Swarm fusion variants in
% Fig.~\ref{fig:model} with restrictions.
%
% 2 k-way 1 solution sharing. \chen{?}
