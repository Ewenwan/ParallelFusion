\section{Related work}


Markov Random Field (MRF) inference has been a very active field in
Computer Vision with extensive literature. We refer the readers to
survey articles for a comprehensive
review~\cite{middlebury_mrf,comparative_study_of_modern_inference}, and
here focus our description on closely related topics.

\mysubsubsection{Parallel Alpha-Expansion}

\noindent Lempitsky et al. introduces parallel computation to an
alpha-expansion technique, where multiple threads simultaneously fuses
mutually exclusive sets of labels. Delong et al.~\cite{delong} and
Veksler~\cite{olga} proposed an hierarchical fusion algorithm, where
labels are simultaneously fused from the bottom to the top in a tree of
labels. Strictly speaking, these parallelize alpha-expansion and are not
in the family of Fusion Move methods, because they never generate
solution proposals. Our approach parallelize fusion move methods.


\mysubsubsection{Parallel MAP inference}

\noindent
The core MAP inference algorithms themselves can be parallelized.
Strandmark et al. parallelized graph-cuts~\cite{strandmark2010parallel}.
%
Message passing algorithms are friendly to GPU implementation and can
exploit the power of parallel computation.
%
However, the core optimization libraries are often very complex and
require significant engineering by experts for modification.
%
While state-of-the-art optimization libraries are often freely available
for non-commercial purposes, most companies have to develop and maintain
in-house implementation of these algorithms. The modification of the
core optimization libraries is a significant engineering investment for
many companies. In contrast, our idea is extremely simple and instantly
reproducible by standard engineers.


% \footnote{GPU speeds-up message-passing algorithms via parallel
% computation. However, these algorithms need to store all the messages
% and states and cannot handle problems with a large label
% space~\cite{layered_depthmap}.}


\mysubsubsection{Fusion Move Methods}

\noindent FM was first introduced by Lempitsky et al. in solving an
optical flow problem~\cite{first_fusion_viktor}. FM has been effectively
used to solve challenging problems in Computer Vision such as a stereo
problem with second order smoothness priors~\cite{woodford}, a stereo
problem with parameteric surface fitting and segmentation (i.e., Surface
Stereo)~\cite{surface_stereo}, or multicut
partiioning~\cite{fusionmovesforcorrelationclustering}.
%
FM has two main advantages over the other general inference
techniques~\cite{trw,loopy_belief_propagation}. \yasu{Pushmeet, is this
correct? Am not confident to say this?} First, FM allows us to exploit
domain-specific knowledge in customizing proposal generation
schemes. Second, FM can handle problems with very large label spaces,
because the core optimization just solves a binary decision problem. In
contrast, methods like message passing algorithms need to maintain
messages and beliefs for the entire label space all the time. 
%
Although conceptually straightforward, we are not aware of {\it parallel
fusion move} algorithms that fuses solution proposals, as opposed to
labels, in parallel. This paper seeks to fully unleash the power of
parallel computation in FM via a more general framework.


\mysubsubsection{Genetic algorithms and Particle Swarm Optimization}

\noindent
Genetic algorithms (GA)~\cite{ga} and Particle Swarm
Optimization (PSO)~\cite{pso} maintain multiple solutions and improve
them over time.
%
While GA and PSO have limited theoretical justification, they have been
used to produce great empirical results, for example, hand tracking via
PSO~\cite{pushmeet_hand_tracking}. At high level, our strategy is
similar in spirit. However, GA or PSO rather arbitrarily copies part of
the solution or makes random movements in each step. Our approach
directly optimizes the objective function to improve solutions.


