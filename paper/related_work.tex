\section{Related work}

FM was first introduced by Lempitsky et al. in solving an optical flow
problem~\cite{first_fusion_viktor}, where proposals

Hiroshi had a paper for doing this for binary variables but not for multilabel variables.

http://www.f.waseda.jp/hfs/CVPR2014.pdf

 This is a good idea.  Rather than having one perfect solution threads, we maintain multiple solution hypothesis which are then combined with each other.  At a high level, this strategy is similar to methods like genetic algorithms (GA) or particle swarm optimization (PSO) where multiple solutions are maintained and improved over time.

 

However, unlike GA or PSO where the fusion step is essentially just arbitrarily copying part of the solution, we try to find the optimal copy.

 

While PSO and GA have limited theoretical justification, they have been
used to produce great empirical results. For instance, in our recent
work on hand tracking, we used the particle swarm optimization
method. See Model Fitting section on page 5 of the following paper:

\url{http://research.microsoft.com/en-us/um/people/pkohli/papers/skrts_chi2015.pdf}

Also see the video here: \url{https://www.youtube.com/watch?v=R-SfOJyZlRQ&list=PLE7aZMY3Wubx6yRgP8U6GRwJCOi2Zyh2S&index=9}

 



 
Note that one may rather seek to parallelize the core optimization
libraries such as Graph-cuts~\cite{}, TRW~\cite{kolmogorov}, or
QPBO~\cite{}.~\footnote{GPU speeds-up message-passing algorithms via
parallel computation. However, these algorithms need to store all the
messages and states and cannot handle problems with a large label space~\cite{layered_depthmap}.} However, the core optimization libraries are often
very complex and require significant engineering by experts for
modification.
%
While state-of-the-art optimization libraries are often freely available for
non-commercial purposes, most companies have to develop and maintain
in-house implementation of these optimization algorithms.
