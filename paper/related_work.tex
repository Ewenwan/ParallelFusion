\section{Related work}


Hiroshi had a paper for doing this for binary variables but not for multilabel variables.

http://www.f.waseda.jp/hfs/CVPR2014.pdf

 This is a good idea.  Rather than having one perfect solution threads, we maintain multiple solution hypothesis which are then combined with each other.  At a high level, this strategy is similar to methods like genetic algorithms (GA) or particle swarm optimization (PSO) where multiple solutions are maintained and improved over time.

 

However, unlike GA or PSO where the fusion step is essentially just arbitrarily copying part of the solution, we try to find the optimal copy.

 

While PSO and GA have limited theoretical justification, they have been used to produce great empirical results. For instance, in our recent work on hand tracking, we used the particle swarm optimization method. See $B!H(BModel Fitting$B!I(B section on page 5 of the following paper:

http://research.microsoft.com/en-us/um/people/pkohli/papers/skrts_chi2015.pdf

Also see the video here: https://www.youtube.com/watch?v=R-SfOJyZlRQ&list=PLE7aZMY3Wubx6yRgP8U6GRwJCOi2Zyh2S&index=9

 

