\section{Conclusion and future directions}
We have proposed a novel MAP inference framework, Swarm Fusion, for MRF
problems in parallel computing environments. The framework is general,
and we showed that most popular inference techniques such as
alpha-expansion, Fusion Move, parallel alpha-expansion, or hierarchical
fusion, are its special cases. Our experiments have revealed that the
framework exploits parallel computational resources and achieves faster
convergence, epseically for challenging problems.  Our first future work
is to conduct experiments on cloud computing environments, in
particular, the MapReduce programming model, where the roles of mappers
and reducers exactly correspond to the processes of parallel multi-way
fusion and solution exchanges, respectively.  Another future work is the
automatic configuration of the Swarm Fusion architecture. Due to its
vast degrees of freedom, one needs to currently hand-tune the
architecture for the optical performance. An interesting direction is to
adaptively change its architecture during the computation, for example,
switching to simple parallel alpha-expansion for easy problems, or
increasing the rate of solution exchanges when solutions vary
significantly across threads.
%
Parallel MAP inference has been a relatively under-explored topic in
Computer Vsion. Our idea is very simple and adapting the Swarm Fusion
framework require minimal coding. We believe that this paper has a
potential to immediately benefit tens of thousands of Computer Vision
researchers or engineers in the world, who currently solve MRF problems
with conventional solvers. We will share our source code and the
datasets with the community.
