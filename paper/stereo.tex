\section{Swarm Fusion instantiation}
% \hang{I didn't understand the organization of this section at the
%   first glance. Should it be clearer if we put 'Swarm fusion
%   architectures' in front of 'Competing methods' in each problem?}

We compare SF against competing approaches over three problems in
Computer Vision, specifically, stereo, optical flow, and layered
depthmap estimation (see Fig.~\ref{fig:problem}).

% SF framework has high degrees of freedom and the
%challenge is to properly quantify the contributions of various
%algorithmic aspects.

%For this purpose, we have used the three SF architectures illustrated in
%Fig.~\ref{fig:model}.
%% For some problems, we have further limited the capabilities of SF
%% on purpose, to enable effective and fair comparative
%% evaluations. We now explain the details of the three problems.

% We pick three representative problems in Computer Vision to evaluate the
% effectiveness of the Fusion swarm methods against existing methods. We
% control the SF architectures for each problem independently to make
% compa
%
%\yasu{this section is very rough at the moment. requires polishing. but
%algorithms might change as we run experiments.}  We now explain detailed
%implementation of the Swarm Fusion method for three problems in Computer
%Vision (See Fig.~\ref{fig:problem}).  5
\begin{figure}[tb]
  \includegraphics[width=\columnwidth]{figure/problem.pdf} \caption{We
    compare our Swarm Fusion method against competing approaches on the
    depthmap stereo~\cite{middlebury_stereo}, the optical
    flow~\cite{middlebury_optical_flow} and the layered depthmap
    estimation~\cite{layered_depthmap} problem. In the layered
    depthmap problem, the input is a RGBD image, and the output is
    multiple layers of depthmaps. Each layer is a piecewise smooth
    parametric surface model.}\label{fig:problem}
\end{figure}
%
%
%
\subsection{Swarm Fusion stereo}
We start with a simple depthmap stereo problem with standard unary and
pairwise terms. We employ submodular pairwise terms to make this
stereo represent relatively ``easy'' MRF inference problem.
%
The unary terms are computed as the average robust photoconsistancy
score~\cite{second_order_stereo} between the reference image and the others
inside a $7\times 7$ pixels window. The pairwise terms are simple
truncated absolute label difference with maximum label difference
$\sigma_s=4$. The total energy is defined by the sum of the two, while
scaling the pairwise terms by a factor of $0.005$. For simplicity we
do not enforce the visibility constraint.

%

\mysubsubsection{Competing methods}

\noindent For simple stereo problems with submodular energy as ours, the
sophistication of photometric consistency
function~\cite{mvs_furukawa_survey} makes unary terms highly
informative, where efficient inference algorithms such as graph-cuts
exist.
%Our experiments have also supported this, where the fusion method
%(i.e., the use of proposal generation) rather makes it slow due to the
%overhead, while not improving final energy.
Therefore, we have chosen algorithms based on Alpha-Expansion, namely
single thread Alpha Expansion(AE), Parallel Alpha
Expansion(PAE)~\cite{fusion_moves_for_markov_random_field_optimization}
and Hierarchical Fusion(HF)~\cite{olga_hierarchical_alpha_expansion} to
be competing methods. For HF, we use Alpha-Expansion at the leaf node of
the label tree and QPBO in the other cases.
%when either of the
%child node is leaf node (constant label) and QPBO for all other nodes.

\mysubsubsection{Swarm Fusion architectures}

\noindent The three swarm architectures in Fig.~\ref{fig:model} have
been evaluated: SF-MF (SF without multi-way fusion), SF-SS (SF without
solution sharing), and the standard SF.
%
SF-MF implies $\alpha+\beta=1$, where each thread repeats
fusing a solution proposal ($\alpha=1, \beta=0$) for four iterations
by Graph-cuts and fusing a concurrent solution  ($\alpha=0, \beta=1$)
for one iteration by QPBO.
%expand 4 labels (repeat $\alpha=1, \beta=0$ for 4 iterations) with Alpha
%Expansion and then fuse one solution from other threads ($\alpha=0,
%\beta=1$) by QPBO.
In the later case, a thread randomly chooses one solution from the
solution pool for fusion. SF-SS implies $\beta=0$, where $\alpha$ is
the free parameter and set to 4. In this case one thread will fuse 4
labels, together with current solution in that thread by TRW-S in each
iteration and never exchanges solutions with other threads. We perform
a multi-way fusion of solutions from all the threads at the end to
obtain a final solution
% at
% the end to fuse solutions from all the threads to a single solution
(similar to PAE). For standard SF architecture, we have used
($\alpha=4, \beta=1$).  To make the comparison simple, we restrict our
solution proposals to be constant-label proposals.
%
% Due to the lack of multi-way fusion, the swarm fusion architecture for
% the stereo problem is represented by the middle example at the bottom of
% Fig.~\ref{fig:model}.
