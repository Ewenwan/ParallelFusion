\section{Swarm Fusion implementations \yasu{I don't like this title..}}
We compare our Swarm fusion methods against competing approaches over
three problems in Computer Vision. The SF architecture has high degrees
of freedom and the challenge is to properly quantity the contribution of
various algorithmic aspects. For this purpose, we carefully design the
SF architecture for each problem independently, sometimes limiting the
capabilities on purpose, to enable effective and fair comparative
evaluation.  We now explain the details of the three problems.

% We pick three representative problems in Computer Vision to evaluate the
% effectiveness of the Fusion swarm methods against existing methods. We
% control the SF architectures for each problem independently to make
% compa
%
%\yasu{this section is very rough at the moment. requires polishing. but
%algorithms might change as we run experiments.}  We now explain detailed
%implementation of the Swarm Fusion method for three problems in Computer
%Vision (See Fig.~\ref{fig:problem}).  5
\begin{figure}[tb]
 \includegraphics[width=\columnwidth]{figure/problem.pdf} \caption{We
 compare our Swarm Fusion methods against competing approaches on a
 depthmap stereo~\cite{middle_bury_stereo}, an optical
 flow~\cite{middlebury_optical_flow}, or a layered depthmap
 estimation~\cite{layered_depthmap} problem. In the layered depthmap
 problem, the input is an RGBD image, and the output is multiple layeres
 of depthmaps. Each layer is a piecewise smooth parametric surface
 model.}\label{fig:problem}
\end{figure}



\subsection{Swarm Fusion stereo}
We start with a simple depthmap stereo problem with standard unary and
pairwise terms. We employ submodular pairwise terms to make this stereo
represent relatively ``easy'' MRF inference problems.
%
Unary terms are computed as the average Normalized Cross Correlation
score between the reference image and the others inside a $7\times 7$
pixels window. Pairwise terms are simple Pottz model. The total energy
is defined by the sum of the two, while scaling the pairwise term by a
factor of \yasu{X value}.
%
We use Middlebury stereo datasets for the experiments.

\mysubsubsection{Competing methods}

\noindent The sophistication of photometric consistency
function~\cite{mvs_furukawa_survey} makes unary terms dominant in most
stereo problems.  Our experiments have also supported this, where the
fusion method (i.e., the use of proposal generation) rather makes it slow
due to the overhead.  Therefore, we pick parallel
alpha-expansion~\cite{delong} and hierarchical fusion~\cite{delong,olga}
with graph-cuts optimization as the competing methods.

\mysubsubsection{Swarm fusion architectures}

\noindent The three swarm architectures in Fig.~\ref{fig:model} have
been evaluated: SFWSS (SF without solution sharing), SFWMF (SF without
multi-way fusion), and the standard SF.
%
For SFWSS, we have chosen $\alpha=2$ ($\beta=0$). For SFWMF, each thread
iterates the step of ($\alpha=1, \beta=0$) or ($\alpha=0, \beta=1$) so
that it can take a solution proposal in one step and a neighboring
solution in the next. For the standard SF, we have used ($\alpha=1, \beta=1$).
%
%
%
%
To make the comparison simple, we restrict our proposal
generation to be constant-label proposals. Graph-cuts is used in fusing
a constant-label proposal as the energy is submodular. QPBO is used in
fusing a different solution from another thread as the energy is
non-submodular.
%
% Due to the lack of multi-way fusion, the swarm fusion architecture for
% the stereo problem is represented by the middle example at the bottom of
% Fig.~\ref{fig:model}.
