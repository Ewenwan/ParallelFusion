\section{Swarm Fusion instantiation}
We compare SF against competing approaches over three problems in
Computer Vision. SF framework has high degrees of freedom and the
challenge is to properly quantify the contributions of various
algorithmic aspects. For this purpose, we have used the three SF
architectures illustrated in Fig.~\ref{fig:model}.
%% For some problems, we have further limited the capabilities of SF
%% on purpose, to enable effective and fair comparative
%% evaluations. We now explain the details of the three problems.

\hang{Better to explain the methology we used, either here or in
  Result part. For example:}

\hang{To evaluate the effectness of different parallel structures, we
  plot the energy minimization process against time. We define the
  energy of a parallel system at a certain time as the minimum energy
  of all threads at that time.}


% We pick three representative problems in Computer Vision to evaluate the
% effectiveness of the Fusion swarm methods against existing methods. We
% control the SF architectures for each problem independently to make
% compa
%
%\yasu{this section is very rough at the moment. requires polishing. but
%algorithms might change as we run experiments.}  We now explain detailed
%implementation of the Swarm Fusion method for three problems in Computer
%Vision (See Fig.~\ref{fig:problem}).  5
\begin{figure}[tb]
  \includegraphics[width=\columnwidth]{figure/problem.pdf} \caption{We
    compare our Swarm Fusion method against competing approaches on a
    depthmap stereo~\cite{middlebury_stereo}, an optical
    flow~\cite{middlebury_optical_flow}, or a layered depthmap
    estimation~\cite{layered_depthmap} problem. In the layered
    depthmap problem, the input is an RGBD image, and the output is
    multiple layeres of depthmaps. Each layer is a piecewise smooth
    parametric surface model.}\label{fig:problem}
\end{figure}



\subsection{Swarm Fusion stereo}
We start with a simple depthmap stereo problem with standard unary and
pairwise terms. We employ submodular pairwise terms to make this
stereo represent relatively ``easy'' MRF inference problems.
%
Unary terms are computed as the average robust photoconsistancy
score~\cite{second_order_stereo} between the reference image and the others
inside a $7\times 7$ pixels window.  Pairwise terms are simple
truncated absolute label difference with maximum label difference
$\sigma_s=4$. The total energy is defined by the sum of the two, while
scaling the pairwise term by a factor of $0.005$. For simplicity we
don't enforce visibility constraint.

%

\mysubsubsection{Competing methods}

\noindent The sophistication of photometric consistency
function~\cite{mvs_furukawa_survey} makes unary terms dominant in most
stereo problems.  Our experiments have also supported this, where the
fusion method (i.e., the use of proposal generation) rather makes it
slow due to the overhead.\hang{Don't quite understand this
  sentence. Does the ``dominant'' means running time or energy? In my
  experiments, unary terms are pre-computed and are not counted into
  the running time} Therefore, we have chosen single thread Alpha
Expansion(AE), Parallel Alpha
Expansion(PAE)~\cite{fusion_moves_for_markov_random_field_optimization}
and Hierarchical
Fusion~\cite{delong_hierarchical_fusion,olga_hierarchical_alpha_expansion}
with graph-cuts optimization as the competing methods.



\mysubsubsection{Swarm Fusion architectures}

\noindent The three swarm architectures in Fig.~\ref{fig:model} have
been evaluated: SF-SS (SF without solution sharing), SF-MF (SF without
multi-way fusion), and the standard SF.
%
SF-SS implies $\beta=0$, where $\alpha$ is the free parameter and set
to 4. In this case one thread will fuse 4 labels by TRW-S per
iteration and never exchanges solutions with other threads. SF-MF
implies $\alpha+\beta=1$, where each thread will expand 3 labels
(repeat $\alpha=1, \beta=0$ for 3 iterations) and then fuse one
solution from other threads ($\alpha=0, \beta=1$) by QPBO. In the later
case, a thread randomly chooses one target thread to grab solution
from. For standard SF, we have used ($\alpha=4, \beta=1$). For the
SF-SS architecture, we performed a final fuse of solutions from all
thread by multiway fusion.

%
%
%
%
To make the comparison simple, we restrict our proposal generator
in each thread to only generate constant-label proposals within a
subset of all labels. Graph-cuts is used in fusing a constant-label
proposal as the energy is submodular. QPBO is used in fusing a
different solution from another thread as the energy is typically
non-submodular.
%
% Due to the lack of multi-way fusion, the swarm fusion architecture for
% the stereo problem is represented by the middle example at the bottom of
% Fig.~\ref{fig:model}.
