\section{Swarm Fusion implementations} \yasu{this section is very rough
at the moment. requires polishing. but algorithms might change as we run
experiments.}
We now explain detailed implementation of the Swarm Fusion method for
three problems in Computer Vision.

\subsection{Swarm Fusion stereo}
We start with a relatively ``easy'' problem, a depthmap stereo problem
with unary and submodular pairwise terms.
%
We use Middlebury stereo datasets for the experiments. Unary terms are
computed as the average Normalized Cross Correlation score between the
reference image and the others inside a $7\times 7$ pixels
window. Pairwise terms are simple Pottz model. Unary terms are rescaled
by a factor of \yasu{X value?} to define the total energy.


\mysubsubsection{Baseline methods}

\noindent
Parallel alpha-expansion~\cite{delong} or hierarchical
fusion~\cite{delong,olga} with Graph-cuts (when submodular) or QPBO
(when non-submodular) have been the most effective parallel techniques.
%

\mysubsubsection{Swarm fusion implementation}

\noindent
To make the comparison simple, we restrict our proposal generation to be
constant-label proposals, and limit to binary fusion so that Swarm
Fusion will also use either Graph-cuts or QPBO for the core
optimization. Limiting to binary fusion deprived SF of multi-way fusion
and SF architecture becomes the middle example in the bottom row of
Fig.~\ref{fig:model}.
%
The advantage of SF is that threads exchange solutions throughout the
process.
