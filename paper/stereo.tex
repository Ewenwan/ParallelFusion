\section{Swarm Fusion implementations}
We compare our Swarm fusion methods against competing approaches over three
problems in Computer Vision. The SF architecture has high degrees of
freedom and the challenge is to properly quantity the contribution of
various algorithmic aspects. For this purpose, we carefully design the
SF architecture for each problem independently, sometimes limiting the
capabilities on purpose, to enable effective and fair comparative evaluation.
We now explain the details of the three problems.

% We pick three representative problems in Computer Vision to evaluate the
% effectiveness of the Fusion swarm methods against existing methods. We
% control the SF architectures for each problem independently to make
% compa
%
%\yasu{this section is very rough at the moment. requires polishing. but
%algorithms might change as we run experiments.}  We now explain detailed
%implementation of the Swarm Fusion method for three problems in Computer
%Vision (See Fig.~\ref{fig:problem}).  5
\begin{figure}[tb]
 \includegraphics[width=\columnwidth]{figure/problem.pdf} \caption{We
 compare our Swarm Fusion methods against various baselines on a
 depthmap stereo~\cite{middle_bury_stereo}, an optical
 flow~\cite{middlebury_optical_flow}, or a layered depthmap
 estimation~\cite{layered_depthmap} problem.  }\label{fig:problem}
\end{figure}



\subsection{Swarm Fusion stereo}
We start with a simple depthmap stereo problem with standard unary and
submodular pairwise terms. Unary terms are computed as the average
Normalized Cross Correlation score between the reference image and the
others inside a $7\times 7$ pixels window. Pairwise terms are simple
Pottz model. The total energy is defined by the sum of the two, while
the pairwise term is scaled by a factor of \yasu{X value}.
%
We use Middlebury stereo datasets for the experiments.

\mysubsubsection{Baseline methods}

\noindent
Parallel alpha-expansion~\cite{delong} or hierarchical
fusion~\cite{delong,olga} with Graph-cuts (when submodular) or QPBO
(when non-submodular) have been the most effective parallel techniques.
%

\mysubsubsection{Swarm fusion implementation}

\noindent
To make the comparison simple, we restrict our proposal generation to be
constant-label proposals, and limit to binary fusion so that Swarm
Fusion will also use either Graph-cuts or QPBO for the core
optimization. Limiting to binary fusion deprived SF of multi-way fusion
and SF architecture becomes the middle example in the bottom row of
Fig.~\ref{fig:model}.
%
The advantage of SF is that threads exchange solutions throughout the
process.
