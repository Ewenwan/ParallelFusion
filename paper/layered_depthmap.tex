\subsection{Swarm Fusion layered depthmap estimation}
Lastly, we will evaluate Swarm fusion on a very challenging layered
depthmap estimation problem~\cite{layered_depthmap}, which has been
recently proposed (see the anonymized pdf in the supplementary
material). The problem seeks to infer layered depthmap representation
from an RGBD image, where the challenge is the massive solution space,
which is exponential in the number of layers.~\footnote{Authors have
proposed a novel fusion scheme, where a solution subspace instead of a
single solution is generated by a proposal generation scheme in each
iteration. Since a solution subspace is eqivalent to multiple solution
proposals, their algorithm can be easily integrated into our swarm
fusion framework.  However, competing fusion methods (e.g., fusion move
or hierarchical fusion) cannot handle a solution subspace proposal,
making it impossible to conduct comparative evaluations. Therefore, we
choose to use a simple solution proposal instead of a subspace in our experiments.}

\mysubsubsection{Baseline methods}
The number of labels is extremely large (an order of 10,000 \yasu{what
were the number of labels in our cvpr16?}), which eliminates the
possibility of using hierarchical fusion. Therefore, we will evaluate
fusion move and parallel fusion move algorithm as in the case of optical
flow problem, where one solution proposal repeats being fused via QPBO.


\mysubsubsection{Swarm fusion implementation}
All the three swarm fusion methods are tested.
