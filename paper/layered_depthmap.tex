\subsection{Swarm Fusion layered depthmap estimation}
Our last problem is layered depthmap estimation, recently proposed
in~\cite{layered_depthmap} (see the anonymious paper in the
supplementary material).  The problem seeks to infer layered depthmap
representation from an RGBD image, where each layer is a piecewise
smooth segmented depthmap. This is essentially a multi-layer extension
of Surface Stereo algorithm~\cite{surface_stereo}.
%
Layered depthmap estimation is one of the most difficult kinds for MAP
inference due to its massive solution space, where the number of labels
per pixel is exponential in the number of layers. In practice, each
pixel might take an order of 100,000 \yasu{is this number correc?}
possible labels. We copy their problem formulation and the proposal
generation schemes.~\footnote{Authors have proposed a novel fusion scheme,
where a solution subspace instead of a single solution is generated by a
proposal generation scheme. Since a solution subspace is equivalent to
multiple solution proposals, their algorithm can be easily integrated
into our swarm fusion framework.  However, competing fusion methods
(e.g., Parallel Fusion Move or Hierarchical Fusion) cannot handle a
solution subspace proposal, making it impossible to conduct fair
comparative evaluations. We choose to use a simple solution proposal for
experiments in this paper.}
%Lastly, we will evaluate Swarm fusion on a very challenging layered
%depthmap estimation problem~\cite{layered_depthmap}, which has been
%recently proposed (see the anonymized pdf in the supplementary
%material).

\mysubsubsection{Competing methods}

Solution proposals depend heavily on the current solution, eliminating
the possibility of using hierarchical approach, which needs to enumerate
all the proposals to start. Therefore, the only viable competing method
is Parallel Fusion Move as in the optical flow problem. We will also
evaluate the standard Fusion Move method.
%
The fusions are binary, for which we use QPBO.

%The number of labels is extremely large (an order of 10,000 \yasu{what
%were the number of labels in our cvpr16?}), which eliminates the
%possibility of using hierarchical fusion. Therefore, we will evaluate
%fusion move and parallel fusion move algorithm as in the case of optical
%flow problem, where one solution proposal repeats being fused via QPBO.


\mysubsubsection{Swarm fusion architectures}
The three swarm architectures in Fig.~\ref{fig:model} are evaluated.
The architecture parameters are exactly the same as in the optical flow problem:
$\alpha=2$ ($\beta=0$) in SFWSS, the iteration of
($\alpha=1, \beta=0$) and ($\alpha=0, \beta=1$) in SFWMF, and
($\alpha=1, \beta=1$) for the standard SF.
%
The fusions are multi-way, for which we use TRW-S.
