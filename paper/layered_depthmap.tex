\subsection{Swarm Fusion layered depthmap estimation}

Our last problem is layered depthmap estimation, recently proposed
in~\cite{layered_depthmap} (see the anonymous paper in the
supplementary material).  The problem seeks to infer layered depthmap
representation from an RGBD image, where each layer is a piecewise
smooth segmented depthmap. This is essentially a multi-layer extension
of Surface Stereo algorithm~\cite{surface_stereo}.
%
Layered depthmap estimation is one of the most difficult kinds in MRF
inference due to its massive solution space. The number of labels per
pixel is exponential in the number of layers, and is usually between 100,000 and 10,000,000. We copy their problem
formulation and the proposal generation schemes.~\footnote{Authors have
proposed a novel fusion scheme, where a solution subspace instead of a
single solution is generated by a proposal generation scheme. Since a
solution subspace can be represented by a concatenation of multiple solution proposals,
% \chen{It is a good explanation for this paper's purpose, but from the perspective of our previous work, I am reluctant to say a subspace equals to multiple solution proposals. When we design proposals, we find the subspace directly and there's no concept of solution proposals (we later simplify our subspace proposals to binary labels just for the comparison reason). It is essentially the same, but a subspace proposal is conceptually more general than concatenated solution proposals.},
their algorithm can be easily integrated into our swarm fusion framework.
However, competing fusion methods (e.g., Parallel Fusion Move or
Hierarchical Fusion) cannot handle a solution subspace proposal, making
it impossible to conduct fair comparative evaluations. We choose to use
a simple solution proposal for experiments in this paper.}
%Lastly, we will evaluate Swarm fusion on a very challenging layered
%depthmap estimation problem~\cite{layered_depthmap}, which has been
%recently proposed (see the anonymized pdf in the supplementary
%material).

\mysubsubsection{Competing methods}

\noindent Solution proposals depend heavily on the current solution,
eliminating the possibility of using Hierarchical Fusion, which needs to
enumerate all the proposals to start. Therefore, viable competing
methods are Fusion Move and Parallel Fusion Move.
%
%We will also evaluate the standard Fusion Move method.
%
The fusions are binary, for which we use QPBO.

%The number of labels is extremely large (an order of 10,000 \yasu{what
%were the number of labels in our cvpr16?}), which eliminates the
%possibility of using hierarchical fusion. Therefore, we will evaluate
%fusion move and parallel fusion move algorithm as in the case of optical
%flow problem, where one solution proposal repeats being fused via QPBO.


\mysubsubsection{Swarm fusion architectures}

\noindent The three swarm architectures with the same configurations as
in the optical flow problems have been evaluated.
%in Fig.~\ref{fig:model} are evaluated.
%The architecture parameters are exactly the same as in the optical flow problem:
%$\alpha=2$ ($\beta=0$) in SFWSS, the iteration of
%($\alpha=1, \beta=0$) and ($\alpha=0, \beta=1$) in SFWMF, and
%($\alpha=1, \beta=1$) for the standard SF.
%
%The fusions are multi-way, for which we use TRW-S.
