\section{Introduction}
Parallel computation has changed the field of computing.  In the 90s,
most processors had single cores. In 2016, most processors have multiple
cores, often 4 or even 8. Cluster computing further expands the
potential of parallel computation, where one can easily launch a
processing job using hundreds or even thousands of computational nodes
in a cloud.
%
Parallel computation offers huge potential for Computer Vision, as
image sensing technologies have gone through revolutions and we are in
ever growing demands in solving very large problems. One may need to
apply image denoising to 50 Megapixel images from latest digital
SLRs~\cite{canon?}, stitch thousands of images to generate gigapixel
panoramas~\cite{gigapan}, or solve volumetric reconstruction and
segmentation problems over a billion ($=1024^3$)
voxels~\cite{Joint_3D_Scene_Reconstruction_and_class_Segmentation}.


Unfortunately, state-of-the-art algorithms are still inherently
sequential in many Computer Vision problems. Take a Fusion-Move
algorithm (FM)~\cite{viktor,second_order_stereo,else} for Markov Random
Field inference, for example, which has been one of the most effective
techniques in solving the above three
problems~\cite{fusion_moves_for_markov_random_field_optimization}.
%
FM was first introduced by Lempitsky et al. in solving an optical flow
problem~\cite{first_fusion_viktor}, where proposals are fused
sequentially one by one to improve the solution. FM allows
domain-specific customization of proposal generation schemes, a main
advantage over general solvers such as message passing
algorithms~\cite{TRW,loopy_belief_propagation}.  Successful applications
of FM range from optical flow, stereo, image denoising, image
inpainting, to image
segmentation~\cite{fusion_moves_for_markov_random_field_optimization}.


Unleashing the power of parallel computation in FM
% Breaking the sequential nature of FM
would then bring fundamental contributions to Computer
Vision. Currently, FM suffers from a few vital limitations due to its
sequential processing. First, standard FM allows only two options per
variable in each fusion, either the current solution or a
proposal~\cite{fusion_moves_for_markov_random_field_optimization}. Second,
only a single proposal generation scheme is used in each fusion
step.~\footnote{Recently, an extension of FM was proposed for layered
depthmap estimation~\cite{chen_2016}, where a solution subspace, instead
of a single solution, is proposed and fused with the current
solution. However, this approach is also limited to the use of one proposal
generation scheme in each fusion.}
%
Our approach, dubbed {\it Parallel Fusion Moves method} (PFM), makes a
few key distinctions from existing approaches: 1) Multiple threads (or
computing nodes) simultaneously keep and improve solutions; and 2) Each
fusion in each thread can take arbitrary number of proposals from
arbitrary combination of proposal generation schemes, even concurrent
solutions in the other threads, to be fused with the current solution.
%
Note that one may rather seek to parallelize the core optimization
libraries such as Graph-cuts~\cite{}, TRW~\cite{kolmogorov}, or
QPBO~\cite{}.~\footnote{GPU speeds-up message-passing algorithms via
parallel computation. However, these algorithms need to store all the
messages and states and cannot handle problems with a large label space~\cite{layered_depthmap}.} However, the core optimization libraries are often
very complex and require significant engineering by experts for
modification.
%
While state-of-the-art optimization libraries are often freely available for
non-commercial purposes, most companies have to develop and maintain
in-house implementation of these optimization algorithms.
%
% \yasu{Mention that existing fm only does binary solution fusion. We does
% multi-solution fusion. This overlaps withour cvpr16, but is probably ok
% by citing the paepr.}
%
Our approach is extremely simple in concept, and will instantly benefit
every Computer Vision researcher or engineer, who currently uses fusion
move algorithm to solve challenging problems. We will share our
implementation of the algorithm. The rest of the paper is organized as
follows. Section~\ref{section:related_work}....
%This paper will propose a novel Parallel Fusion
%Move method, which fuses solution proposals in parallel. Our framework
%can exploit parallelism from the levels of threading to the levels of
%computer nodes.


%The main limitation of FM is its running time, where Computer Vision
%researchers and engineers generally consider FM to be a slow algorithm
%that can solve challenging problems, as hundreds or sometimes thousands
%of proposals need to be fused sequentially.
%


%\yasu{Are there any concrete examples where MRF is used to solve a very
%large problem where parallel fusion is really helpful. Like denoising
%high resolution 24M pixel images. Stitching a high resolution map tiles
%(satellite image tiles) via image stitching.}

%
%This paper will propose a {\it Parallel Fusion Move} (PFM) method that
%introduces the concept of parallel fusion into the computational
%framework.



%
 \yasu{Hopefully, more interesting things will come up through
experiments and we have more to say. For example, it would be cool if
uniform model is better than pure master-slave. Then we can write that
ours is not just parallel. But keeping multiple solutions somehow
benefit (GA/GP algorithm's ideas).}


%\yasu{mention GPU somewhere? strange to talk about parallel without
%mentioning GPU and its relation to our approach}
