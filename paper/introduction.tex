\section{Introduction}
\hang{Should we use $\alpha$-expansion instead of alpha-expansion?}

Parallel computation has changed the field of computing.  In the 90s,
most processors had single cores. In 2016, most processors have multiple
cores, often 4 or even 8. Cluster computing further expands the
potential of parallel computation, where one can easily launch a
processing job using hundreds or even thousands of computational nodes
in a cloud.
%
In the recent work on the AI program playing the ancient Chinese board
game of {\it Go}, parallelization plays a key role in the Monte-Carlo
tree search~\cite{nature_alpha_go}.


Parallel computation offers tremendous potential for Computer Vision. As
image sensing technologies have gone through revolutions, we are in
ever growing demands in solving very large problems. One may need to
apply image denoising to 50 Megapixel images from latest digital
SLRs (e.g., Canon EOS 5DS), stitch thousands of images to generate gigapixel
panoramas~\cite{gigapan}, or solve volumetric reconstruction and
segmentation problems over a billion ($=1024^3$)
voxels~\cite{Joint3DSceneReconstructionandclassSegmentation}.
%
%
However, state-of-the-art algorithms are still inherently sequential in
many Computer Vision problems. Take a Fusion Move method
(FM)~\cite{viktor,second_order_stereo,else} for example, which has been
one of the most effective techniques for Markov Random Field (MRF)
inference, including the above problems like image denoising, image
stitching, or volumetric reconstruction.
%
It sequentially improves solution by fusing the current solution with a
solution proposal. Its successful applications go beyond and also cover
optical flow, stereo, image inpainting, or image segmentation
problems~\cite{fusion_moves_for_markov_random_field_optimization}.


Unleashing the power of parallel computation for effective MRF inference
% Breaking the sequential nature of FM
would then bring fundamental contributions to Computer
Vision. Currently, FM suffers from a few vital limitations due to its
sequential nature. First, standard FM allows only two options per
variable in each fusion, either the current solution or a
proposal~\cite{fusion_moves_for_markov_random_field_optimization}. Second,
only a single proposal generation scheme is used in each fusion
step.~\footnote{Recently, an extension of FM was proposed for layered
depthmap estimation~\cite{chen_2016}, where a solution subspace, instead
of a single solution, is proposed and fused with the current
solution. However, this approach is also limited to the use of one proposal
generation scheme in each fusion.}
%
Our approach, dubbed {\it Swarm Fusion} method (SF), makes a
few key distinctions from existing approaches: 1) Multiple threads (or
computing nodes) simultaneously keep and improve solutions; and 2) Each
fusion in each thread can take arbitrary number of proposals from
arbitrary combination of proposal generation schemes, even concurrent
solutions in the other threads, to be fused with the current solution.
%


We have evaluated the effectiveness of our approach over three problems
in Computer Vision, stereo, optical flow, and layered
depthmap estimation.
%
%Our approach outperforms all the other competing methods, especially for
%challenging problems with a large label space.
Our idea is very simple. We believe that the method is easily
reproducible with minimal coding, and would have immediate impact on
numerous Computer Vision researchers or engineers, who currently solve
MRF problems.
\chen{We will share our implementation of our pipeline with clear interface,
  so that others can utilizing the parallel pipeline with minimal coding}
% will instantly benefit every Computer Vision researcher or
% engineer, who currently uses fusion move algorithm on multi-core
% environments to solve challenging problems.
% We will share our implementation of the algorithm.


%
% \yasu{Mention that existing fm only does binary solution fusion. We does
% multi-solution fusion. This overlaps withour cvpr16, but is probably ok
% by citing the paepr.}
%
%This paper will propose a novel Parallel Fusion
%Move method, which fuses solution proposals in parallel. Our framework
%can exploit parallelism from the levels of threading to the levels of
%computer nodes.


%The main limitation of FM is its running time, where Computer Vision
%researchers and engineers generally consider FM to be a slow algorithm
%that can solve challenging problems, as hundreds or sometimes thousands
%of proposals need to be fused sequentially.
%


%\yasu{Are there any concrete examples where MRF is used to solve a very
%large problem where parallel fusion is really helpful. Like denoising
%high resolution 24M pixel images. Stitching a high resolution map tiles
%(satellite image tiles) via image stitching.}

%
%This paper will propose a {\it Parallel Fusion Move} (PFM) method that
%introduces the concept of parallel fusion into the computational
%framework.



%
%  \yasu{Hopefully, more interesting things will come up through
% experiments and we have more to say. For example, it would be cool if
% uniform model is better than pure master-slave. Then we can write that
% ours is not just parallel. But keeping multiple solutions somehow
% benefit (GA/GP algorithm's ideas).}


%\yasu{mention GPU somewhere? strange to talk about parallel without
%mentioning GPU and its relation to our approach}



%1. In the introduction, we could mention the recent work on Go (DeepMind's AlphaGo Nature paper) in which parallization plays a key role in the monte carlo search.

%2. Yasu, your explanation of the advantages of Fusion Move (described in
%section 2) is correct. In the sentence on "Second, FM can handle
%problems with very large label spaces.." we could add "and even
%real-valued variables".

%3. I would suggest removing "Genetic algorithms" from the paragraph heading at the end of section 2 or replacing it with "Evolutionary Algorithm".

%4. We should cite two more paper
