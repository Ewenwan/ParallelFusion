\section{Introduction}
Parallel computation has changed the field of Computing.  In the 90s,
most processors had single cores. In 2016, most processors have multiple
cores, often 4 or even 8. Cluster computing further expands the
potential of parallel computation, where one can easily launch a
processing job using hundreds or even thousands of computers in a
cloud. Computer Vision is not an exception in exploiting the powerful
parallel computation. However, in many Computer Vision problems,
state-of-the-art algorithms are still inherently sequential. Take a
Fusion-Move algorithm (FM)~\cite{viktor,second_order_stereo,else} for
example. FM was first introduced by Lempitsky et al. in solving an
optical flow problem. FM allows one to customize domain-specific
proposal generation schemes to achieve efficient convergence. FM has
been a very successful optimization framework in solving many
challenging problems in Computer Vision such as optical
flow~\cite{viktor}, stereo~\cite{second_order_stereo}, or image
segmentation~\cite{some}. The main limitation of FM is its running time,
where Computer Vision researchers and engineers generally consider FM to
be a slow algorithm, as hundreds or sometimes thousands of proposals
need to be fused sequentially. This paper will propose a {\it Parallel
Fusion Move} (PFM) method that introduces the concept of parallel fusion
into the computational framework.

FM suffers from a few fundamental limitations due to its sequential
nature. First, standard FM offers only two options in each fusion,
either the current solution or a single proposal. Recently, an extension
of FM was proposed for layered depthmap estimation~\cite{chen_2016},
where a solution subspace, instead of a single solution, is proposed
from each proposal generation scheme. However, a solution is
sequentially updated by a single generation scheme one by one. In our
framework, multiple threads (or computing nodes) simultaneously keep and
improve solutions. Furthermore, in each fusion step in each thread, our
approach can take any number of proposals from any number of proposal
generation schemes, and even concurrent solutions in the other threads
to fuse them into a new solution.
%
Note that one can seek to parallelize the core optimization libraries
such as Graph-cuts~\cite{}, TRW~\cite{kolmogorov}, or QPBO~\cite{} to
achieve similar speed-up. However, these core optimization libraries are
often very complicated and used as a complete black box. Some libraries
are provided only in a binary form, especially commercial
packages. Furthermore, this library modification cannot benefit from
higher level parallelism through distributed computer nodes.
\yasu{Mention that existing fm only does binary solution fusion. We does
multi-solution fusion. This overlaps withour cvpr16, but is probably ok
by citing the paepr.}
%
The idea in this paper is extremely simple and easily reproducible. This
paper can instantly benefit every Computer Vision researcher or
engineer, who currently uses fusion move algorithm to solve challenging
problems. We will share our implementation of the algorithm. The rest of
the paper is organized as follows. Section~\ref{section:related_work}....
%This paper will propose a novel Parallel Fusion
%Move method, which fuses solution proposals in parallel. Our framework
%can exploit parallelism from the levels of threading to the levels of
%computer nodes.
%
 \yasu{Hopefully, more interesting things will come up through
experiments and we have more to say. For example, it would be cool if
uniform model is better than pure master-slave. Then we can write that
ours is not just parallel. But keeping multiple solutions somehow
benefit (GA/GP algorithm's ideas).}

